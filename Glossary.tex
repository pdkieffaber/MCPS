%GLOSSARY TERMS
\newglossaryentry{transduction}{name={transduction},description={The conversion of physical energy (light, sound, etc.) to action potentials}}
%-----------------------------------------------
%        Data Collection Glossary
%-----------------------------------------------
\newglossaryentry{open source software}{
type={datacollection},
name={Open Source Software},
text={open source software},
description={Open source software is software with source code that anyone can inspect, modify, and enhance.}}

\newglossaryentry{polling rate}{type={datacollection}, name={Polling Rate},
text={polling rate},description={The rate at which the status of a peripheral input to a computer like a keyboard or mouse is checked.}}

\newglossaryentry{measurement error}{type={datacollection}, name={Measurement error},text={measurement error},description={Measurement Error is the difference between a measured quantity and its true value. Measurement error can be due to random errors of measurement or systematic errors such as a mis-calibrated measurement instrument.}}

\newglossaryentry{URL}{type={datacollection}, name={URL},description={A Uniform Resource Locator (URL), colloquially termed a web address,[1] is a reference to a web resource that specifies its location on a computer network and a mechanism for retrieving it.}}

\newglossaryentry{web hosting service}{
    type={datacollection}, 
    name={Web hosting service},
    text={web hosting service},
    description={A web hosting service is a type of Internet hosting service that allows individuals and organizations to make their website accessible via the World Wide Web. Web hosts are companies that provide space on a server owned or leased for use by clients, as well as providing Internet connectivity, typically in a data center.}}
 
\newglossaryentry{platform independent}{
    type={datacollection},
    name={Platform independent},
    text={platform independent},
    description={describes software that does not depend on a specific operating system like Windows MacOSx, or Linux.}
}

\newglossaryentry{Compile}{
    type={datacollection},
    name={Compile},
    text={compile},
    description={Convert a program or script into machine-code or lower-level form in which the program can be executed by a computer.}
}

\newglossaryentry{principal investigator}{
    type={datacollection},
    name={Principal Investigator (PI)},
    text={principal investigator},
    description={The lead researcher for a grant project, laboratory study, or a clinical trial. The phrase is also often used as a synonym for ``head of the laboratory" or ``research group leader".}
}

\newglossaryentry{aggregate data}{
    type={datacollection},
    name={Aggregate Data},
    text={aggregate data},
    description={High-level data which is acquired by combining individual-level data.}
}

\newglossaryentry{task schematic}{
    type={datacollection},
    name={Task Schematic},
    text={task schematic},
    description={A figure depicting the major components, sequence and, and timing of an experimental task.}
}

\newglossaryentry{inter-stimulus interval}{
    type={datacollection},
    name={Inter-stimulus Interval (ISI)},
    text={inter-stimulus interval},
    description={The interval of time between the offset of one stimulus and the onset of the next stimulus in an experiment.}
}

\newglossaryentry{inter-trial interval}{
    type={datacollection},
    name={Inter-trial Interval (ITI)},
    text={inter-trial interval},
    description={The interval of time between the offset of a stimulus marking the end of an trial and the onset of a stimulus marking the beginning of the next trial in an experiment.}
}

\newglossaryentry{extraneous variables}{
    type={datacollection},
    name={Extraneous variable},
    text={Extraneous Variable},
    description={Extraneous variables are all variables, which are not the independent variable, but could affect the results of the experiment.}
}

\newglossaryentry{variable}{
    type={datacollection},
    name={Variable},
    text={variable},
    description={Variables in programming are like containers that hold information for later reference or manipulation. Their purpose is to label and store data.}
}

\newglossaryentry{conditional execution}{
    type={datacollection},
    name={Conditional Execution},
    text={conditional execution},
    description={When one or more instruction(s) are executed only when a specific test condition is determined to be true.}
}

\newglossaryentry{informed consent}{
    type={datacollection},
    name={Informed Consent},
    text={informed consent},
    description={Informed consent is a process for getting permission from participants before conducting an intervention conducting any form of research on a person, or for disclosing a person's information.}
}

\newglossaryentry{local variable}{
    type={datacollection},
    name={Local Variable},
    text={local variable},
    description={In computer science, a local variable is one that is defined in a function or block and that is available only in that function or block. It is not accessible outside the block.}
}

\newglossaryentry{global variable}{
    type={datacollection},
    name={Global Variable},
    text={global variable},
    description={A variable whose value is accessible from inside any function or block.}
}


%-----------------------------------------------------------------------------------
%.    GRAPH THEORY
%----------------------------------------------------------------------------------

\newglossaryentry{undirected graphs}{
    type={graphtheory},
    name={Undirected Graphs},
    text={undirected graphs},
    description={A graph with bidirectional relationships.}
}

\newglossaryentry{Directed graphs}{
    type={graphtheory},
    name={Directed Graphs},
    text={Directed graphs},
    description={A graph with unidirectional relationships.}
}

\newglossaryentry{Degree}{
    type={graphtheory},
    name={Degree},
    text={Degree},
    description={A fundamental graph metric that counts the number of edges connected to a node. In the example graph, each node has degree 2, meaning it connects to exactly two other nodes in the hexagonal arrangement. For directed graphs, degree can be split into in-degree (incoming connections) and out-degree (outgoing connections).}
}

\newglossaryentry{centrality}{
    type={graphtheory},
    name={Centrality},
    text={centrality},
    description={A fundamental graph metric that counts the number of edges connected to a node. In the example graph, each node has degree 2, meaning it connects to exactly two other nodes in the hexagonal arrangement. For directed graphs, degree can be split into in-degree (incoming connections) and out-degree (outgoing connections).A measure of a node's importance or influence within a network.}
}

\newglossaryentry{clustering coefficient}{
    type={graphtheory},
    name={Clustering Coefficient},
    text={clustering coefficient},
    description={A measure of the degree to which nodes in a graph tend to cluster together, quantifying how close a node's neighbors are to being a complete graph.}
}
